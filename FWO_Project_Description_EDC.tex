% Compile with XeTeX, e.g. latexmk -xelatex -shell-escape -pvc -pdf report.tex
% Template based on Ruben Verborgh's FWO template.
\documentclass[a4paper,11pt]{article}

% Page setup
\usepackage[a4paper,margin=3cm,right=2.5cm]{geometry}
\usepackage{changepage}

% Typography
\usepackage{fontspec}
\setmainfont{Calibri}
\usepackage{microtype}
\usepackage{csquotes}
% Paragraphs
% Sections
\usepackage[noindentafter]{titlesec}
\titleformat\section{\bfseries}{}{0em}{}
\titlespacing\section{0em}{1em}{0em}
\titleformat\subsection{\fontsize{13pt}{13pt}\selectfont\bfseries}{}{0em}{}
\titlespacing\subsection{0em}{1em}{.5em}
\titleformat\subsubsection{\color{black!65}\bfseries}{}{0em}{}
\titlespacing\subsubsection{0em}{0.66em}{0.13em}
\titleformat\paragraph[runin]{\bfseries\itshape\color{black!60}}{}{0em}{}
\titlespacing\paragraph{0em}{.5em}{.66em}
% Lists
\usepackage[inline]{enumitem}
\setlist{nosep}
% Varia
\newcommand\expl[1]{\textcolor{black!50}{\emph{#1}}}
\usepackage{comment}

% References
\usepackage[hidelinks]{hyperref}
\usepackage[capitalize,nameinlink,noabbrev]{cleveref}
\let\UrlFont\itshape
\usepackage[
  backend=biber,
  bibstyle=trad-plain, % Was trad-abbrv, but that does not compile
  abbreviate=false,
  doi=false,
  isbn=false,
  giveninits=true,
  sorting=none,
  sortcites=true,
  citestyle=numeric-comp,
]{biblatex}
\renewcommand\multicitedelim{\addcomma}
% Custom author list length
\addbibresource{EDC_references.bib}
\renewcommand*\bibfont{\fontsize{10pt}{13pt}\selectfont}
\bibitemsep 1pt plus 1pt minus 1pt
% Slanted "et al."
\usepackage{xpatch}
\xpatchbibmacro{name:andothers}{%
  \bibstring{andothers}%
}{%
  \bibstring[\textsl]{andothers}%
}{}{}
% Underline my name
\usepackage[normalem]{ulem}
\def\firstname{Elias}
\def\lastname{Crum}
\renewcommand\mkbibnamegiven[1]{%
  \ifboolexpr{test {\ifdefequal\firstname\namepartgiven} and
              test {\ifdefequal\lastname\namepartfamily}}
  {\uline{\namepartgiveni~\namepartfamily}}
  {\namepartgiveni~\namepartfamily}%
}
\renewcommand\mkbibnamefamily[1]{}

% \strong
\makeatletter
\ifdefined\strong
\renewcommand{\strong}[1]{\@strong{#1}}
\else
\newcommand{\strong}[1]{\@strong{#1}}
\fi
\newcommand{\@@strong}[1]{\textbf{\let\@strong\@@@strong#1}}
\newcommand{\@@@strong}[1]{\textnormal{\let\@strong\@@strong#1}}
\let\@strong\@@strong
\makeatother
% Set contributions apart from other bolding
\newcommand{\contribution}[1]{\strong{\textcolor{RoyalBlue}{#1}}} % but a less ugly color?

% Reviewing
% \input{reviewing}

% Graphics
\usepackage[framemethod=TikZ]{mdframed}
\usepackage{pgfgantt}

% Misc
\hyphenation{LTBQP IBQP}

\begin{document}

\begin{mdframed}[backgroundcolor=black!17,linecolor=black!0,font=\bfseries]
	\centering
	PHD FELLOWSHIP STRATEGIC BASIC RESEARCH\\
	PROJECT OUTLINE (MAX. 12 A4 pages)\\
	\end{mdframed}
	\vspace{-.5\baselineskip}
\title{PHD FELLOWSHIP STRATEGIC BASIC RESEARCH PROJECT OUTLINE}

\begin{refsection}

\section{Rationale and positioning with regard to the state-of-the-art}
% Elaborate the scientific motivation for the project proposal based on scientific knowledge gaps, and the issues and/or problems that you want to solve with this project. Concisely describe the related international state of the art, with reference to scientific literature. Position your project in relation to ongoing national and international research.
% --- aim for 2 pages ---

\smallskip

\paragraph{Decentralized Landscape}
% decentralization to combat centralization (include examples?)
Data decentralization initiatives~\cite{solid, mastodon, decentralizednanopubs} are working to reduce the data siloing caused by data centralization on the Web.
A leading decentralized storage strategy is the use of \textbf{personal data vaults}.
The Solid protocol in particular offers user-moderated access controls, data linking in and across vaults using the Resource Description Framework (RDF)~\cite{spec:rdf}, represented as triples with universal semantics, built on Linked Data principles~\cite{linkeddata}, and information extraction via querying using the SPARQL query language~\cite{spec:sparqllang}.
Implementations of Solid for industry use are emerging~\cite{athumi}, but other challenges presented by data querying infrastructure and governance remain.


\paragraph{Personal Genome Sequencing in Healthcare}
% why could genome sequencing benefit from decentralization
% current centralization limits scalability --> due to costs
% costs are high because PGS data  is big, expensive to generate, and requires a lot of maintenance oversight
% cost decreasing measures (sharing/data linking/interoperability?) are hard due to privacy constraints (PGS data is sensitive)
% centralized database technologies pit privacy and cost reduction as antagonists...

Around the same time that the World Wide Web was being established, DNA sequencing technologies were just being applied to the human genome~\cite{hood_1987}.
At the time of writing, there are now multiple domains of clinical practice where patient \textbf{personal genome sequence (PGS) data are now used to inform medical decision making}. 
Examples include in drug development~\cite{ko_new_2022}, cancer diagnosis and treatment~\cite{mcleod_cancer_2013}, and rare genetic disease identification and treatment~\cite{souche_recommendations_2022}.
How this integration is deployed varies by clinical domain, but improved outcomes have generally been observed~\cite{mathur_personalized_2017}.
Despite great promise presented by various use cases, barriers to widespread adoption remain~\cite{stefanicka-wojtas_barriers_2023}.

\textbf{One major barrier to scalability is presented by the costs of PGS data generation and storage}~\cite{genomics_cost_2023}.
The average human genome is slightly over 3 billion base pairs in length and during a whole genome sequencing workflow, various sequence formats that offer different sets of information are produced~\cite{bagger_whole_2024}.
Of these, Variant Call Format (VCF) files~\cite{danecek_variant_2011} serve as the state-of-the-art for most clinical genomic applications and are \textbf{typically between 100-1000s MBs} within computer memory. 
Because of their large size, there are existing state-of-the-art methods for compressing~\cite{vcf_compression_2022} and parsing~\cite{yang_seqminer2_2020} these files using indexes.

The costs of producing and maintaining these data are further increased by the \textbf{privacy protections needed for PGS data}~\cite{GDPR_2016}.
With the enlarged threat of hacking, phishing, and login credential compromisation that is only increasing~\cite{ransomware}, health care institutions have taken steps to enact tighter restrictions on data access and increase cyber security budgets.
Because of this data siloing, there is \textbf{little to no data sharing between health care institutions}. 
If a patient moves hospitals, it is common for PGS data and genomic tests to be regenerated and indefinetly stored in that new location because of the lack of data sharing infrastructure.
To reduce costs and improve the scalability of PGS usage in clinical practice, alternative methods of data storage and privacy protection that allow for improved data sharing are merited.
\emph{I will establish user-friendly methods for the management and protection of stored PGS data while also making it possible to share that data without compromising its privacy.}


% \paragraph{PGS data sharing in academic research}
% Because hospital systems have a wealth of clinically relevant data, but that data is largely inaccessible outside of those institutions, research motivated initiatives working to increase the accessibility of that data have gained traction recently~\cite{data_sharing_2019}.
% Practically, this has led to the development of infrastructure that allows for sharing of genome data between institutions, creating federated centralized databases. 
% Among the largest and most prevalent in Europe is GA4GH Beacons~\cite{rambla_beacon_2022}. 
% Despite this step towards increased sharing and cost reduction, advancements in state-of-the-art infrastructure and standards are not directly translatable to clinical practice. 


\paragraph{Decentralized PGS data storage}
The \textbf{citizen-centric model} places the patient at the center of their data, and is not an entirely novel concept~\cite{brands_patient-centered_2022}.
Within the current system, a citizen-centric model is difficult to implement due to technological challenges presented by centralized, relational databases.
The \textbf{Solid protocol}~\cite{solid}, a decentralized data storage approach, is composed of specifications \textbf{more conducive to construction of a citizen-centric data storage strategy} for clinical data.
Specifically, Solid offers the ability to \textbf{granularize data privacy policies}, allow \textbf{authorized data access over the web}, and represent stored data as \textbf{Linked Data}~\cite{linkeddata}, all features that can work to remove some of the antagonism between cost reduction and privacy preservation.
In recent years, there have been initiatives for representing biological data as RDF \cite{sib_rdf_2024}, specifically extending into clinical biology recently~\cite{sphn_rdf_2023}. 
While there is little research into the benefits of representing genomic data as RDF, past studies have shown that \textbf{linked data integration into clinical practice results in improved outcomes}~\cite{farinelli_linked_2015}.
Furthermore, using Solid Pods for data storage also makes it possible for non-linked data, such as test result files, to be linked to RDF data, improving data connectivity.
As of yet, \textbf{decentralized storage technologies} have not meaningfully been used in clinical practice.
If implemented, they could provide improved data sharing, reduced data duplication, and increased data privacy controls that could \textbf{contribute to clinical PGS cost reductions and improved scalability}.
Adaptation of Solid decentralized technology to clinical genomics does not come without challenges.
Serious challenges are posed by the large size of genomic data, the interoperability of future storage technologies with current applications and tools, and the lack of existing infrastructure for implementation of a decentralized storage framework.  
To drive innovation in the field of health data storage,~\emph{I will create and implement workflows for creating, hosting, and using Solid Pods for the storage of patient genomic and clinically relevant data.}


\paragraph{Decentralized Data Querying} 
% Implementation challenge:
% Solid --> no way to discover data (because no computation)
% Solution --> Need a source of computation to query data
% decentalized data storages, such as Solid, are not intrinsically endowed with having access to computation to perform querying
% So, query engines are used to perform queries over this data.

A challenge of Solid personal data vaults is that they do not intrinsically have access to computation. 
Thus, \textbf{to read data within a Solid pod, a query engine is needed for accessing and parsing stored data}.
Query engine implementations within Solid are an ongoing area of research, and one framework for research-based implementions is provided by Comunica~\cite{comunica}.

There are also \textbf{algorithmic challenges presented by querying genomic data vaults}.
The processing of queries can be characterized by two stages, the planning and the execution. 
The traditional strategy followed a sequential optimize-then-execute approach, where query planning is done first based on pre-existing dataset statistics then the produced plan guides execution. 
For cases where there are not dataset statistics available, adaptive processing has been presented as a solution, where planning and execution are recursively performed throughout the querying process~\cite{adaptive_book}.

SPARQL endpoints~\cite{spec:sparqlprot}, large centralized well-indexed databases, represent examples of environments where query planning and execution are guided by indexes to improve performance~\cite{sparqlqueryoptimization}.
Federated querying algorithms~\cite{hibiscus,tpf} build on those approaches for querying over multiple, but not a large number of, SPARQL endpoints.

For querying decentralized ecosystems, challenges are presented by the larger number of sources and inconsistency of indexes or statistics about those sources available prior to execution.
Due to these constraints, recent work presenting Link Traversal Query Processing (LTQP) algorithms has established new approaches for federated querying in these decenralized environments~\cite{taelman_LTQP_2023}.
For additional efficiency and specialized use cases, the use of aggregators that store summaries representing data types or other statistics about sources has also been proposed~\cite{VandenbrandeAggregatorsTR, taelman_privacyAgg_2020}.

The established LTQP algorithms assume conditions where \textit{many sources contain small amounts of data}, which is different than those presented by patient genome pods.
In the case of genomic data pods, querying will be performed over \textit{a potentially large number of data pods containing large amounts of linked data}, a situation not extensively investigated. 
\textbf{In this context, it is likely that existing LTQP algorithms and query planning strategies will require innovation.} 
\emph{I will develop new LTQP algorithms that are capable of querying deventralized genomic data vaults through the use of genome indexes and data vault summaries.}

\paragraph{Towards clinical implementation}
Despite there being no real solutions to the current antagonism between privacy and cost reduction for PGS data usage in health care, there is also a conspicuous gap in the current scientific discourse around the development and implementation of a proposed solution. 
This gap underscores the necessity of my Ph.D.
I aim to improve the \textbf{connectedness and shareability of genomic data storage(s), while preserving data privacy}, through the integration of cutting edge of semantic web research in the domains of \textbf{data storage, policies, and querying}.
\emph{I will accomplish this by creating a novel, holistic framework that is implemented as a web application, complete with guides and documentation to foster usability, for use in clinical practice.}
Through this framework, I will also demonstrate the limitations of current state-of-the-art semantic web technologies in this novel application domain to drive innovation and uncover future research pursuits.


\section{Scientific research objective(s)}
% --- aim for 1 page ---

\begin{comment}
Describe explicitly the scientific objective(s) and the research hypothesis. 
Explain whether and how the research is specifically challenging and inventive, describing in particular the innovative aspects of the envisaged results. 
Discuss in detail the results (or partial results) that you aim to achieve, such as specific knowledge, the solution to particular problems and academic breakthroughs.
\end{comment}
\smallskip

\noindent
My proposed research endeavors to combine cutting edge decentralized storage technology with semantic data representation and federated querying technologies into a novel proof-of-concept PGS data storage and querying framework for use in clinical practice. 
To create such a framework, I will combine technologies from different distinct areas of semantic web research and apply them to a data ecosystem that poses novel challenges.
This ambition frames the central research question I aim to answer: \textbf{Can combining the Solid specifications for data storage with other compatable cutting edge innovations in data policy, linking, and querying be instantiated and deployed as a framework that provides clear advantages over the existing PGS data storage protocols in health care?}

The core research question can be decomposed into four more specific research questions.
\textit{First, can the decentralized storage protocol Solid}~\cite{solid} \textit{offer suitable infrastructure for PGS data?} 
I hypothesize that the Solid protocol will be able to store clinical genomic data.
Further, I aim to establish Solid also offers usage advantages over existing systems through the representation of \textbf{PGS data using RDF as Linked Data}~\cite{berners-lee_linked_2009}.
A further aim within this objective is exploring if storage of PGS data as RDF using \textbf{Header Dictionary Triples (HDT) format}~\cite{hdt} provides similar levels of usability of genomic data with significantly \textbf{decreased storage costs}.
To ensure the widest range of connection capabilities while optimizing efficiency, I will also investigate the use of a \textbf{bi-directional mapping index} for the conversion between native genomic VCF files and RDF representations.
For these aims, my background knowledge of genomic file anatomies and how data is semantically represented in genomics file formats will be valuable, specifically when optimizing indexing and format conversion strategies.

Second, because of the sensitive nature of PGS data, \textit{do the specifications provided by Solid provide for adequate control of PGS data privacy while also allowing for increased authorized sharability?}
I aim to demonstrate possible configurations of Solid data vault privacy policies as well as offer a functionality within a \textbf{web application for the alteration of these policies} by authorized users.
I am uniquely situated for assessing these privacy policies because of my past experience working in a clinical setting alongside physicians as well as my understanding of different sets of US and EU regulations mediating data privacy requirements.

Third, for the stored genomic data and linkages to be usable in clinical practice, a querying method is necessary.
In a citizen-centric clinical data storage implementation, there could be potentially thousands of large sources to be queried over.
Performant federated decentralized querying in such environments is an established challenge~\cite{dang_fedshop_2023}.
Therefore, \textit{can querying over these sources be achieved through the use of LTQP algorithms?}
\textbf{I anticipate that current generalized LTQP algorithms will not be able to perfrom well over large genomic data}.
Thus, I will investigate two strategies for improving the performance of query processing.
(A) I will investigate the \textbf{use of summaries} of patient data vault contents, stored outside of data vaults in aggregators, to modify query planning strategies in LTQP algorithms.
Because of the privacy considerations inherint with patient data vault contents, I will follow presviously descibed theoretical methods to maintain data privacy in these summaries~\cite{taelman_privacyAgg_2020}.
(B) I will develop LTQP algorithms that \textbf{integrate genomic data indexes} for within-vault querying.
I have experience with index guided genomic data parsing which will help inform developing and optimizing query processing algorithms to incorporate these guides. 

Together, these components will be \textbf{combined into an operational framework} in the fourth component. 
The driving question being, \textit{can these three different groups of features be combined into a cohesive web application and deployed together?}
The framework, once produced, will be compared to existing strategies for storing and sharing PGS data to assess the efficacy of transitioning toward product production and specific clinical use case adaptation.
The proposed scientific approach also aims to test the application of numerous fields of semantic web research to a clinical knowledge domain. 
In the process, insight into how unique challenges introduced by clinical constraints will provide future areas of research, both applied and fundamental.

\newpage

\section{Research methodology and work plan}
% --- aim for 4 pages ---

\begin{comment}
Elaborate the different envisaged steps (experiments/activities) in your research, and motivate strategic choices in view of reaching the objectives. 
Describe the set-up and cohesion of the work packages including intermediate goals (milestones).
Show where the proposed methodology (research approach) is according to the state of the art and where it is novel. 

Discuss risks that might endanger reaching project objectives and the contingency plans to be put in place should this risk occur.
Use a table or graphic representation of the planned course of activities (timing work packages, milestones, critical path) over the 4-years grant period.
\end{comment}

\medskip

\noindent
My research plan consists of three component objectives, representative of three core functionalities of my proposed framework.
A final fourth component will be the unification of the three functional components into a web application for deploying the framework.
%
First, I focus on the foundational infrastructure for data storage and formatting for the framework.
%
Second, I focus on framework data privacy policies for granular, flexible data policy enforcement.
%
Third, I integrate querying functionality to the data storage framework using a query engine approach and modified LTQP algorithmic approach to allow for data discoverability.
%
After each component, I elaborate on its risks.
Last, I present my work plan.

\newcommand\WPa{Storage and formatting PGS data in a citizen-centric architecture}
\subsection{Component~1: \WPa}

The foundation of my proposed framework is the data storage infrastructure.
To increase the efficiency of data storage and usage, a citizen-centered data storage approach will be attempted.
This organizational strategy is not feasable in centralized databases given current technologies, thus, I will utilize a decentralized storage approach.
Of the decentralized storage initiaves~\cite{solid, mastodon, decentralizednanopubs}, Solid was chosen because it is not social network specific, is growing in popularity, and has specifications useful for privacy and data sharing infrastructure.

%maybe put an additional task here for review paper composition??

\newcommand\WPaa{Storing PGS data in Solid data vaults}
\subsubsection{Task~1.1: \WPaa}

% Using my experience working with genomic data, I will try to use solid to efficiently store PGS files
% I will show that PGS data storage is possible, thereby shoing that large data storage in Solid is possible.
% result will be the development of a workflow for creating, hosting, and uploading data into patient solid pods

Here, I will test the viability of Solid data pods for patient PGS data storage, thus, testing my hypothesis that Solid can support PGS data storage. 
Using my experience with genomic data types and file representations, I will assemble a test dataset composed of publicly available genome files~\cite{platinum_genomes, 1000_genomes}. 
These files will be used as representative "patient" PGS data for all future experimentation. 
I will also create server-hosted Solid pods using the Community Solid Server (CSS) implementation of Solid~\cite{css}. 
The use of the CSS for Solid pod hosting for research purposes is state-of-the-art, but there have been no published experiments documenting the use of CSS pod instances for storing PGS data, which are much larger than in past Solid experimentation.
Each pod will be a storage container for a single individual\textquotesingle s PGS data. 
I will upload a single PGS file, a VCF file, into one "patient\textquotesingle s" pod to test basic functionality of a Solid pod for hosting large genomic data. 
The result of this task will be the \textbf{development and documentation of a workflow for creating, hosting, and uploading PGS data into patient solid data vaults}.


\newcommand\WPab{Representing PGS data as Linked Data using RDF}
\subsubsection{Task~1.2: \WPab}

To improve data storage efficiency and future application potential, I will \textbf{convert PGS data from VCF to RDF} allowing for linking of other medically relevant data to patient genomic data within a patient\textquotesingle s pod and outside of it.
This aim will address well documented current challenges in medicial record utilization relating to scatteredness of pertinent clinical information~\cite{pastorino_benefits_2019}.
To convert PGS data from VCF to RDF, we will investigate a format translation process using the SPHN RDF ontology~\cite{van_der_horst_bridging_2023}. 
For this translation process, I will experiment with using a bi-directional mapping index for efficient reversal of conversion to ensure connection to existing clinical workflows that request VCF format inputs. 
Direct conversion between VCF and RDF will be evaluated in terms of computational overhead, conversion time, and memory usage.
Evaluations detailing the use of an intermediate mapping index file will also be done and compared to direct VCF to RDF conversion. 
These comparisons will be documented in a formal benchmarking study.
Because representation of VCF files in RDF has not been heavily studied, these will be the first published experiments of their kind.

Data that is serialized as RDF can be represented in a number of formats in computer memory~\cite{rdf_serializations}.
To minimize the storage costs of large PGS data, I will utilize \textbf{HDT format to compress the PGS data} while retaining the ability to query and index it~\cite{hdt}.
This approach has not be applied to genomic data before.

I then intend to demonstrate the \textbf{linking of part of a patient\textquotesingle s genome to}
\textbf{(A)} other data within the patient\textquotesingle s pod, 
\textbf{(B)} data in a public database outside of a patient\textquotesingle s pod, and
\textbf{(C)} data from another patient\textquotesingle s pod.
The power of linking the VCF data to other clinically relevant data will be during querying, which will be performed in Component 3.
While Linked Data is state-of-the-art, these concepts have not yet been applied to clinical genomic data.


\subsubsection{Risks}
The main risk of storing PGS data in Solid data vaults (Task 1.1) is related to the size of PGS data. 
I will have access to servers at UGhent and VITO NV where implementations of experimental Solid pods will not face size limitations that interfere with project progress.

The main risk of converting PGS data to Linked Data using RDF (Task 1.2) is that this conversion requires an ontology. 
The ontology offers semantic information about the PGS data being stored, thus the specific semantic standards used will be important for future discoverability through querying. 
To make produced RDF data as universally applicable as possible, I will only focus on converting VCF data to RDF, and this will be done using the publicly available SPHN RDF ontology~\cite{sphn_rdf_2023}.
If this ontology is insufficient, I will work with members of the IDLab at UGhent with experience in ontology definition to create my own ontology for the conversion process.


\newcommand\WPb{PGS data privacy policies}
\subsection{Component~2: \WPb}

\newcommand\WPba{Implementing data policies}
\subsubsection{Task~2: \WPba}
A large advantage of the Solid decentralized data storage protocol over current institution-centric methods of data storage is the more flexible methods of creating, modifying, and enforcing data access policies. 
I will experiment with the \textbf{design and implementation of multiple levels of authorization as well as methods that allow for dynamic control over data discoverability, read/write access, and data access consent requests} within a patient\textquotesingle s Solid pod, made possible by the Solid specification. 
I will develop and test three functionalities for privacy modifications.
(1) registration of a pod to an individual patient,
(2) submission of a request to access stored data from a data requester, the notification of the patient, and the consent or denial by the patient, and
(3) permission revoking capabilities as well as an opt-in option to share their data with researchers. 
All of these methods will be integrated into the framework\textquotesingle s web application.
To utilize these methods, various levels of access to pod read and write privileges will be created to fill the needs and roles of participants of a PGS clinical workflow. 
Attaching differing levels of authorization to data will be assessed by creating various profiles that reflect clinical roles and access levels and attempting to access data via user-mediated, application requesting, and querying approaches. 
Assigning the above permissions within Solid is an open area of research and there are currently state-of-the-art protocols implemented in the CSS that allow their implementation~\cite{css}.
The described access schema has not been attempted in the presented level of detail for clinical genomic data.

\subsubsection{Risks}
If the above proposed schema for privacy policies cannot be achieved, a simplier and more generalized schema will be devised and implemented.
Privacy is a nuanced subject especially in terms of governance concerning sensitive data. 
I aim to show the possibilities preseted by Solid in this framework, not dictate suggestions for its deployable implementations. 


\newcommand\WPc{Querying PGS data over one and many data vaults}
\subsection{Component~3: \WPc}

% Most research done with decentralized federated querying has been on many small pods
The problem space presented by citizen-centric genomic data vaults is novel for federated decentralized environment querying. 
Established techniques for federated SPARQL querying over a \emph{small number of large sources} have been documented~\cite{hibiscus, tpf}.
Techniques for federated SPARQL querying over decentralized environments with a \emph{large number of small data sources} have also been documented~\cite{taelman_LTQP_2023}.
Personal genomic data vaults for clinical practice present a third querying landscape consisting of a \emph{large number of large data sources}.
One advantage presented by the PGS data stored in the proposed manner is that assumptions about the data contained within patient data vaults can be made based on the conserved function of vaults.
This homogeneity of data will be leveraged to inform query algorithm planning.
Concretely, \textbf{I will assess and improve LTQP algorithms for genomic data vaults}.
Novel improvements to existing algorithms will be attempted by using within patient vault PGS data indexes and outside patient vault privacy preserving summaries.
Benchmarking studies will assess the performance of these algorithms among various possible clinical usecases.


\newcommand\WPca{Link Traversal Query Processing algorithm benchmarking}
\subsubsection{Task~3.1: \WPca}

This work package will establish a querying mechanism for data in the patient Solid pods that takes into account patient pod data, user permissions, and data linkages. 
To query these linkages, I will utilize the knowledge graph query language SPARQL~\cite{spec:sparqllang}.
Query execution requires a source for computation which is not currently provided by the Solid pods themselves.
I will implement an instance of Comunica~\cite{comunica} to perform the queries.
Because Comunica is open-source and intended for research purposes, it will enable experimentation with modified query processing algorithms. 

For actual benchmarks, \textbf{I will assemble a suit of represetative patient genomic data vaults}.
I will first assemble a test set of 10 patient data vaults to test basic functionalities, then two additional test sets of 100 patient data vaults and 1000 patient data vaults to represent real-world use cases where it is assumed that many pateints\textquotesingle data can be stored by my framework.
Query functionality will be evaluated using query execution time and computational load metrics as well as a query results assessment. 
Query results will additionally establish the functionality of data linkages (Task 1.2).
LTQP algorithms are an active area of active research, but most of the work done has been with generalized algorithms and datastores with small amounts of data~\cite{taelman_LTQP_2023}.
I aim to adapt this querying approach to the specific domain of genomic and health data which has not been attempted before. 

Benchmarking will be \textbf{initially performed for existing LTQP algorithms}~\cite{taelman_LTQP_2023}.
Initially, success will be determined by queries that return correct results verified by a truth-set.
In a clinical setting, time constraints are not as important as accuracy and reliability of results, motivating our primary assessment criteria.
Because excessive query times may decrease the usefulness of such a tool for physicians in clinical practice, I will also assess the time it takes for queries to execute and attempt to minimize this time in the following tasks.

\newcommand\WPcb{Privacy-preserving data vault summaries}
\subsubsection{Task~3.2: \WPcb}

It is \textbf{unlikely that existing LTQP algorithms perform well} for genomic data vaults due to their large size and low connectivity to other patient data vaults.
Therefore, I will experiment with the generation of \textbf{data vault summaries, stored in aggregators, that do not compromise the privacy of patient data}. 
To implement these privacy preserving summaries, I will first assemble within-data vault summaries with designated access controls.
These summaries may be generated using some of the data from genomic bi-directional mapping indexes (Task 1.2).
Then, one or multiple aggregator(s) will assemble multi-vault summaries that are stored outside the data vaults using a summary combination algorithm~\cite{taelman_privacyAgg_2020}.
These summaries are intended to be used by modified LTQP algorithms (Task 3.3) to improve query processing time and efficiency. 
The described privacy preserving summaries have only been proposed in theory and the proposed implementation will be the first of its kind.


\newcommand\WPcc{Algorithm incorporation of PGS data indexes and data vault summaries}
\subsubsection{Task~3.3: \WPcc}

For the optimization of LTQP algorithm performance over PGS data vaults, I will look to improve existing algorithms by \textbf{incorporating previously generated aggregator summaries (Task 3.2) and genome data indexes (Task 1.2)}.
I hypothesize the summaries will be specifically useful for query planning because they can help scope the number of data vaults that need to be queried among other benefits. 
Alongside the use of summaries, I also will experiment with the use of within data vault indexes for the querying of genomic data specifically. 
There are well established VCF file parsing tools that allow for highly performant parsing of VCF data via the use of an indexing strategy~\cite{yang_seqminer2_2020}. 
I indend to implement a similar strategy by using a pre-computed genome index (Task 1.2) to improve the performance of genome-specific queries. 
Combining the two together, a dynamic algorithmic approach that allows for summaries and genome indexes to inform query rewriting at differnt points during query compilation and execution will be examined.  


The algorithms described above will be benchmarked using the same benchmarking set and evaluation criteria as generic LTQP algorithms (Task 3.1). 
I will additionally benchmark the query times of a single data vault PGS to existing tools VCF files parsing tools on the bases of speed, computational load, and result correctness.

The federated querting algorithms utilizing indexes and summaries proposed are novel in nature and have not been developed before. 

\subsubsection{Risks}
Task 3.1 presents the risk that PGS data are too large for LTQP algorithms to execute over. 
If these issues arise, I will assess the execution of simpler queries as well as the execution of multiple, sequential queries over a subset of the genome data vaults.

Another risk is associated with the generation of secure summaries in Task 3.2. 
If these summaries cannot be created in a way that adheres to the privacy demands of PGS data, I will investigate the use of other, more secure methods.
Another option that may be explored is the creation of aggregator data vaults, with privacy protections, where summaries can be stored to improve privacy protections.

Lastly, there is a risk that I cannot devise solutions to incorporating indexes and/or summaries in LTQP algorithms and/or these algorithm modification do not improve performance enough to be usable in Task 3.3.
I will initially investigate how imposing limits on query complexity and reducing the number of data vaults that are included in the possible query space could improve performance.
If there are still issues, I will investigate the implementation of algorithms utilized by centralized SPARQL endpoints that are known to be able to query large sets of data. 


\newcommand\WPd{Ph.D. Finalization}
\subsection{Component 4: \WPd}
\newcommand\WPda{Framework consolidation and deployment}
\subsubsection{Task~4.1: \WPda}
The three components will be combined into a functional framework. 
The framework will include a \textbf{web application that offers a central location for accessing the functionalities discussed above}.
The framework will be deployed and demonstrated as it could be used in clinical practice.

\newcommand\WPdb{Ph.D. dissertation composition and defense}
\subsubsection{Task~4.2: \WPdb}
The findings and results will be packaged into a \textbf{Ph.D. dissertation and defense}.

\subsubsection{Risks}
The main concern is that some of the functionalities or components of previous work packages will not be able to be integrated together into a single web application. 
If necessary, different applications will be created to accommodate any components that do not fit into the planned unified web application.

\newpage
\subsection{Work plan}
\smallskip

\noindent
My project consists of 4 work packages that correspond with the componenets presented above. 


%have not gotten to editing this table yet
\begin{adjustwidth}{0cm}{0cm}
	\parindent 0pt
	\newcommand\duration[1]{\hfill\emph{#1~months}}
  
	\textbf{WP1:      \WPa  \duration{10}}
	\begin{itemize}
	  \item Task 1.1: \WPaa \duration{ 4}
	  \item Task 1.2: \WPab \duration{ 6}
	\end{itemize}
	\smallskip
  
	\textbf{WP2:      \WPb  \duration{8}}
	\begin{itemize}
	  \item Task 2: \WPba \duration{ 8}
	\end{itemize}
  \smallskip
	
	\textbf{WP3:      \WPc  \duration{18}}
	\begin{itemize}
	  \item Task 3.1: \WPca \duration{ 6}
	  \item Task 3.2: \WPcb \duration{ 6}
	  \item Task 3.3: \WPcc \duration{ 12}
	\end{itemize}
	\smallskip
	
	\textbf{WP4:      \WPd  \duration{ 12}}
  \begin{itemize}
	  \item Task 4.1: \WPda \duration{ 6}
	  \item Task 4.2: \WPdb \duration{ 9}
	\end{itemize}
\end{adjustwidth}

\noindent
I will undertake work packages and tasks as shown below in a Gantt chart.
The primary work packages of my Ph.D. are WP1 and WP3.
WP2 is a supporting work package and WP4 will be completed by combining all other work packages.
WP1 and WP3 are dependent, meaning that WP1 must be partially completed before WP3 can begin.
Thus, these work packages are planned sequentially, which provides me with knowledge of how data storage architecture could be leveraged for WP3.

The focus of WP1 is on storing and formatting PGS data.
The development of a workflow for setting up Solid data vaults and getting PGS data into those vaults is the first step (Task 1.1).
Once the PGS data is stored, investigation into data format conversion and data representation will be performed (Task 1.2).

For WP2, I will apply privacy protecting policies to stored PGS data using the Solid protocol (Component 2).

In WP3, I will build on the work of WP1 and assess the query performance of LTQP algorithms on genomic data vaults.
I will start by benchmarking existing LTQP algorithms in a formal benchmarking study (Task 3.1).
To optimize query performance, I will experiment with the production of privacy preserving summaries stored in aggregators outside of data vaults (Task 3.2).
Then, I will develop modified LTQP algorithms that use genomic indexes and data vault summaries to improve query efficiency (Task 3.3).

I will unify the three framework components into a web application (Task 4.1) that will be deployed as a demonstration complete with documentation and user guides. 
Throughout the previous work packages, I aim to publish incremental findings in high-impact journals, such as the Semantic Web Journal and Journal of Healthcare Informatics Research, and conferences, such as the International Semantic Web Conference and the European Conference on Computational Biology.
The completed Ph.D. will be combined into a dissertation and defended in Fall 2027 (Task 4.2).

\noindent
\begin{ganttchart}[
  x unit=8.5pt,
  y unit title=12pt,
  y unit chart=11pt,
  bar height=1,
  bar top shift=0,
  title height=1,
  group height=.1,
  group/.append style={draw=none,fill=none},
  vgrid={black!15},
  hgrid style/.style=black!50,
  bar label font=\it,
  title label font=\bf,
]{1}{48}
  \gantttitle{2023}{ 2}
  \gantttitle{2024}{12}
  \gantttitle{2025}{12}
  \gantttitle{2026}{12}
  \gantttitle{2027}{10}
  \\
  \ganttset{title label font={}}
  \gantttitle{Q4}{2}
  \gantttitle{Q1}{3}\gantttitle{Q2}{3}\gantttitle{Q3}{3}\gantttitle{Q4}{3}
  \gantttitle{Q1}{3}\gantttitle{Q2}{3}\gantttitle{Q3}{3}\gantttitle{Q4}{3}
  \gantttitle{Q1}{3}\gantttitle{Q2}{3}\gantttitle{Q3}{3}\gantttitle{Q4}{3}
  \gantttitle{Q1}{3}\gantttitle{Q2}{3}\gantttitle{Q3}{3}\gantttitle{Q4}{1}
  \\
  \ganttgroup{WP1}{ 1}{8}\\
  \ganttset{bar/.append style={fill={rgb,255:red,220;green, 35;blue,  0}}}
  \ganttbar{T1.1}{1}{ 4}
  \\
  \ganttbar{T1.2}{5}{ 10}
  \\[grid]
  \ganttset{bar/.append style={fill={rgb,255:red, 87;green,157;blue, 28}}}
  \ganttbar{\bf WP2}{11}{18}
  \\[grid]
  \ganttgroup{WP3}{ 19}{36}\\
  \ganttset{bar/.append style={fill={rgb,255:red,  0;green,132;blue,209}}}
  \ganttbar{T3.1}{19}{21}
  \ganttbar{    }{35}{37}
  \\
  \ganttbar{T3.2}{21}{26}
  \\
  \ganttbar{T3.3}{23}{35}
  \\[grid]
  \ganttgroup{WP4}{37}{47}\\
  \ganttset{bar/.append style={fill={rgb,255:red,255;green,149;blue, 14}}}
  \ganttbar{T4.1}{38}{43}
  \\
  \ganttbar{T4.2}{41}{48}
\end{ganttchart}


\section{Strategic dimension and application potential}
% Aim for 1-2 page?
\begin{comment}
Elaborate the strategic dimension of your research, with regard to the 
(long-term) potential for innovative applications. 
Substantiate the PhD project’s strategic focus on economically relevant innovations. 

Justify how the chosen research approach (if successful) is the appropriate one to achieve the anticipated application(s) (potentially long term).

Elaborate the strategic importance of the potential applications to possible users (impact). 
Show how (if the project is successful) new products, services and/or processes may affect business of specific companies, a collective of companies and/or a sector and/or may be closely aligned with the Flemish science, technology and innovation transition priorities  (Flanders in transition. Priorities in Science, Technology and Innovation towards 2025) (socio-economic benefits). 

Societal impact should always be linked to a (in)direct (macro)economic benefit, e.g. cost reductions in health care, higher education level, environmental impact etc. should be positioned in an economic context.
\end{comment}
\smallskip

% personalized medicine --> more sequences
% sequences are expensive and sensitive
% for there to be more personalized medicine, sequence generation and maintenance needs to decrease
% sequence generation cost has been decreasing but storing it has not
% therefore we have a problem to solve
My Ph.D. project is strongly motivated by the potential economic and societal gains presented by personalized medicine. 
For all existing and future applications of genetically-informed precision health, patient genome sequence data in some form will be required.
In recent years, the cost of digital genome sequence generation has steadily decreased~\cite{wetterstrand_cost_2021}, but over that same time the cost of storing and maintaining the privacy of that data has not kept pace~\cite{storage_costs}.
Thus, unintended barriers to scaling current clinical genomic workflows as well as to researching new workflows have been observed.

% there are different ways to solve the above problem but right now they cant happen (largely because of technological constraints)
% my framework allows us to solve the posed privacy <--> sharing problems 
% along with other cool features (connection to products purchased by the consumer / querying for clinicians) that could be built upon in the future
% generally my project is saying "if we store this stuff in a new way there are so many new possibilities for applications"
% this roughly equates to developing a new niche market (or even multiple niche markets)
% examples include: personal genomic products that can use your WGS, data linking to ANY kind of medical data, potential for being financially compensated for allowing your genome to be used for research, etc...
The framework that will result from my Ph.D. will be uniquely positioned to compete with the current state-of-the-art intitiution-centric data storage systems due to the flexibility and cost efficiency it offers as a citizen-centric approach.
The approach accomplishes this cost efficiency though infrastructure for privacy preserving patient genomic data sharing, data policy customization, and integrated data querying capabilities.
Because most data is confined within a single health care institution, data sharing between institutions is an economic niche that is largely unfilled. 

The private genomic service industry dominated by companies such as 23andMe, Ancestry.com, sequencing.com, and others establishes that genomics data generation and storage holds importance to consumers for various personal and medical reasons. 
At the same time, hospital systems exclusively store and maintain all patient PGS data that is used for clinical applications. 
There is notable nuance between these two sectors including different forms of genomic data being generated, stored, and used, differing legal oversight concerning commercial genomic data and health data, and formatting differences between the genomic data stored. 
Regardless, in our modern age of big data, data duplication due to data siloing, energy waste due to computational demands during data regeneration, and intrinsic security concerns for modern data storage techniques are major economic inefficiencies of the current system. 

A hypothetical company that, in coordination with policy makers and regulatory bodies, creates a scalable storage and data sharing infrastructure for genomic data, which could also grow to include all patient health data in time, stands to greatly increase the efficiency of PGS data usage in healthcare. 
% boom Disruptive innovation. 
% New market, data no longer needed to be maintained by hospitals, a new chance for establishing standardization, new market for health data applicaitons, etc.
Such efficiency increases could help lower patient costs for specialized genetic tests, remove data management and administration from hospitals, thereby reducing costs, and establish a new market within which economic growth could result. 

% one of the great advantages of my framework is that it allows for the separation of clinically relevant data from a single health care institution. 
% doubtless, there will need to be regulatory presence in deploying a product version
% but, that is sort of built into the current project
% here I am presenting how the tools for the system COULD be put together and prove that within the constraints presented by health care data they CAN produce a functional system
% if it is directly translated into a project GREAT, but the more realistic goal is to provoke other ideas that improve the presented system
% the goal is to make more personalized medicine more doable, citizen-centric data control is the most attractive economic and ethical path toward that goal.
My project presented above is designed to present a proof-of-concept framework, both providing and demonstrating the technological foundations for the storage of PGS data in Solid pods, the controlling of access to that data on a granular level, the ability for that data to be queried, and exhibiting the accessibility of the stored PGS data to users, web applications, and medical tools in formats that can be used by both those currently in use and applications developed in the future. 
Such a framework will provide the outline of necessary implementation considerations from a technological perspective while also highlighting strengths and weaknesses of such a system that may be influential in attempts at scaling such an infrastructure. 
My project is also being undertaken in parallel with the European Virtual Human Twin (EDITH)~\cite{edith} initiative that aims at evolving the way medical data is stored to be increasingly citizen/patient centric both within Flanders as well as the greater European Union. 
The WE ARE project~\cite{we_are} is another Flemish initiative exploring ways in which citizen-centric data stprage and ownership.


In the short-term, the project is being developed to be integrated into ongoing research and product development at VITO Digital Precision Health. 
Products aimed at improving the way drug prescription is practiced by using a genetic screening tool that leverages documented genetic predispositions to drug ineffectiveness are currently being developed to be connected to my framework of Solid pod stored PGS data. 
Additionally, connection to other known and widely-used workflows such as for NIPT and rare genetic disease screening is a primary goal for my project. 
Genomic data interoperability is of utmost importance for clinical application and is therefore a cornerstone of my project. 

The European Health Data Space (EHDS) aims to establish a reliable, interoperable, and secure environment for health data for all EU citizens~\cite{EHDS_2022}

Lastly, public perception is a crucial element to the economic growth of a product or sector. 
With personal data usage transparency as well as greater calls for digital data privacy protections becoming more important to the public, such considerations should also be priorities to how health data is managed. 
The existing system of genomic data storage for use in healthcare is prone to data leaks and heavily restricted patient transparency due to the central architecture of institution-centric data stores. 
With my proposed framework, patients would be more intimately connected to their data, potentially even having a say over to whom and what their data is visible. 
Such improved transparency, when paired with decreased risk of large-scale data leaks, is likely to be well-received by the general public. 
Such public support could help drive such a framework adoption to a larger scale such as nationally or even to be the standard for a system like the EU. 
This large scale goal, while nowhere near attainable in the near future, would present the greatest possible outcome for such a project and exhibit a somewhat unintuitive increase in greater genomic data privacy and shareability. 
In this scenario, there is also room for healthy competition within such a niche as various pod providers could offer hospital systems and educational institutions different rates for data storage and associated computation.


\section{References}
% Give an overview of the bibliographical references that are relevant for your research proposal. 
\smallskip
\printbibliography[heading=none]


\end{refsection}

\end{document}
