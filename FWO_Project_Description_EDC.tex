\documentclass[a4paper,11pt]{article}
\setmainfont{Calibri}


\usepackage{comment}

\begin{document}

\title{PHD FELLOWSHIP STRATEGIC BASIC RESEARCH PROJECT OUTLINE}

\author{Elias Crum}[
orcid=0009-0005-3991-754X,
email=elias.crum@ugent.be
]

\maketitle

\section{1. Rationale and positioning with regard to the state-of-the-art}
\begin{comment}
Elaborate the scientific motivation for the project proposal based on scientific knowledge gaps, and the issues and problems that you want to solve with this project. Concisely describe the related international state of the art, with reference to scientific literature. Indicate why the execution of the proposed strategic basic research is important. Position your project in relation to ongoing national and international research.
\end{comment}

\subsubsection{Decentralized Landscape}
Data decentralization initiatives [c] are working to reduce the data siloing caused by data centralization on the Web.
A leading decentralized storage strategy are personal data vaults that offer user moderated access controls[c], data linking across vaults using the Resource Description Framework[c], and information extraction via querying using the SPARCL query language[c].
The goals of this initiative is to give individual people control over their data, while increasing data sharability and reusability.

Implementations of this data storage and usage approach include [c].


\subsubsection{Personal Genome Sequencing in Healthcare}
Something to connect DNA sequencing and the Web...
At the time of writing, there are multiple domains of clinical practice where patient PGS data is now used to inform medical decision making. 
Examples include in drug development \cite{ko_new_2022}, cancer diagnosis and treatment \cite{mcleod_cancer_2013}, and rare genetic disease identification and treatment \cite{souche_recommendations_2022}.
How this integration is deployed varies by clinical domain, but improved outcomes have generally been observed \cite{mathur_personalized_2017}.
Despite great promise presented by various use cases, barriers to widespread adoption remain \cite{stefanicka-wojtas_barriers_2023}.

One major barrier to scalability is presented by the costs of data generation and storage[c].
The average human genome is slightly over 3 billion base pairs in length and during a whole genome sequencing workflow, various sequence formats that offer different sets of information are produced \cite{bagger_whole_2024}.
Of these, Variant Call Format (VCF) files \cite{danecek_variant_2011} serve as the state-of-the-art for most clinical genomic applications and are typically between 100-1000s MB (0.1-1 GB) within computer memory. 

Another is the computational costs of regenerating results and sequences because of little to no data sharing potential in the current system.

The costs of producing and maintaining these data are additioanlly increased by the privacy protections needed for PGS data[GDPR].
With the enlarged threat of hacking, phishing, and login credential compromisation that seems to only be increasing \cite{noauthor_ransomware_nodate}, hospitals and institutions are forced to enact tighter regulation over data access within their institution, severely limit outside institution access, and increase their cyber security budget to handle security audits.

\textbf{PGS data sharing in academic research.}
In the realm of academic research, the development of infrastructure that allows for sharing of genome data between institutions, creating federated centralized databases, has gained traction recently. 
Initiatives such as GA4GH Beacons \cite{rambla_beacon_2022} and others are building infrastructure for this between institution data sharing. 
Despite this step towards increased sharing and cost reduction, advancements in state-of-the-art infrastructure and standards are not directly translatable to clinical practice. 


\subsubsection{Decentralized PGS data storage}
A possible solution to the challenges faced is through reorganization of how data is stored and discovered. 
The citizen-centric model places the patient at the center, and is not an entirely novel concept \cite{brands_patient-centered_2022}.
Within the current system, a citizen-centric model is difficult to implement due to technological challenges presented by centralized databases.
The Solid protocol \cite{capadisli_solid_nodate}, a decentralized data storage approach, is composed of specifications more conducive to construction of a citizen-centrica data storage strategy for clinical data.
Specifically, Solid offers the ability to granularize data privacy, allow authorized data access over the web, and represent stored data as Linked Data, all features that can work to remove some of the antagonism between cost reduction and privacy preservation.
In recent years, there have been initiatives for representing biological data as RDF \cite{sib_swiss_institute_of_bioinformatics_rdf_group_members_sib_2024}, specifically extending into clinical biology recently \cite{van_der_horst_bridging_2023}. 
While there is little research into the benefits of representing genomics data as RDF, past experiments have shown that linked data integration into clinical practice results in improved outcomes \cite{farinelli_linked_2015}.

Furthermore, using Solid pods for data storage also makes it possible for non-linked data stored in the pod, such as test result files, to be linked to RDF data, improving data connectivity.
As of yet, decentralized storage technologies have not meaningfully expanded to use in clinical practice.
but if they did, things like data sharing, reduced data duplication, increased data privacy controls, could contribute to the PGS cost reductions and improved scalability.
On the other side of the coin, the size and sensitive nature of PGS data provide a relatively unstudied frontier of decentalized web technology.

\textbf{Link Traversal Query Processing (LTQP)}
To make sense of linked genomic and clinical data, approaches to parsing and querying that data must also be investigated, especially to encourage greater data discoverability and usage in clinical practice.
Recent work has established that the querying of Linked Data in decentralized environments is possible \cite{taelman_evaluation_2023}, but these results were obtained with assumptions different than those presented by patient genome pods.
Here, querying will be performed over a potentially large number of data pods containing large amounts of linked data, a situation not extensively investigated. In this context, it is likely that existing LTQP algorithms and approaches will require innovation. 

It is documented how many personal data vaults of small amounts of data can be queried, but little has been done to investigate how many data vaults of large amounts of relatively similar data could be queried.

At present, the current state-of-the-art methods for data storage are centralized in nature, following an institution-centric model. 
This data storage strategy has posed great challenges to the scaling of personalized medicine, the use of patient genomic information to inform clinical decision-making.


\textbf{Project Motivation.}
Despite there being no real solutions to the current antagonism between privacy and cost reduction for PGS data usage in healthcare, there is also a conspicuous gap in the current scientific discourse around the development and implementation of a proposed solution. 
This gap underscores the necessity of my Ph.D.
I aim to improve the connectedness and shareability of genomic data storage(s), while preserving data privacy, through the integration of various domains of semantic web research into a novel, holistic framework designed for use in clinical practice. 
My Ph.D. will also aim to demonstrate the limitations of current state-of-the-art semantic web technologies in this novel application domain with the intention of driving innovation and discovering future research pursuits.



\section{2. Scientific research objective(s)}
\begin{comment}
Describe explicitly the scientific objective(s) and the research hypothesis. Explain whether and how the research is specifically challenging and inventive, describing in particular the innovative aspects of the envisaged results. Discuss in detail the results (or partial results) that you aim to achieve, such as specific knowledge, the solution to particular problems and academic breakthroughs.
\end{comment}

My research project is situated to provide a proof-of-concept PGS data storage and sharing framework for use in clinical practice. 
In this pursuit, my core research questions are: 
(A) Can a citizen-centric PGS data storage framework, developed using a decentralized storage technology, offer data sharing infrastructure while maintaining privacy safeguards for sensitive data? 
(B) Can PGS data be stored as RDF, be linked to other medically-relevant data, and be queried over in a ways that would be useful to clinical practice?
To address these central questions, I will investigate the use of data pods, implemented using Solid protocol, to store PGS data in a decentralized and privacy-oriented ecosystem while addressing technical challenges associated with data policies, representation of PGS data as Linked Data, discoverability through querying, and infrastructure that connects data pods to existing genomic data sharing initiatives. 
My hypothesis is that such a framework can be developed and would offer unique advantages over the existing state-of-the-art institution-centric PGS data storage solutions. 

The following five objectives will be undertaken to test my hypotheses through experimental production and testing of the listed components of the proposed framework.
(1) Solid Pod PGS Data Storage, 
(2) Genomic data as Linked Data,
(3) Data policies,
(4) Data querying,
(5) Framework deployment.

Together, these objectives will serve as the components of an operational framework. 
The framework, once produced, will be compared to existing strategies for storing and sharing PGS data to assess the efficacy of transitioning toward product production and specific clinical use case adaptation.
The proposed scientific approach also aims to test the application of numerous fields of semantic web research to a clinical knowledge domain. 
Explicitly, an approach to how decentralized storage specifications can be applied to sensitive medical data storage, how genomic and medical data can be represented and queried as Linked Data, how existing Linked Data querying algorithms perform over genomic and health data, how granularized data policies impact querying and linking data in a medical context, and how the combination of these semantic technologies could provide an improvement over existing state-of-the-art clinical PGS data storage and usage strategies, are specific questions my Ph.D. aims to answer.


\section{3. Research methodology and work plan}
\begin{comment}
\textit{Elaborate the different envisaged steps (experiments/activities) in your research, and motivate strategic choices in view of reaching the objectives. Describe the set-up and cohesion of the work packages including intermediate goals (milestones).
Show where the proposed methodology (research approach) is according to the state of the art and where it is novel. Discuss risks that might endanger reaching project objectives and the contingency plans to be put in place should this risk occur.
Use a table or graphic representation of the planned course of activities (timing work packages, milestones, critical path) over the 4-years grant period.}
\end{comment}

My Ph.D. will be split into four component work packages (WPs) with a fifth work package where the components will be unified into a cohesive framework with an accompanying web application. This workflow is reflected in Fig. 1.

\begin{figure}
\includegraphics[width=\textwidth]{fig1.eps}
\caption{\textbf{PENGQUIN Ph.D. workflow.}
White circles represent locations within the workflow where milestones will be achieved. 
For each of the steps in the workflow, the applicable work packages also shown in parentheses. 
The small blue "WWW" box seen next to some objectives represents integration of that item into the framework\textquotesingle s web application.
} \label{fig1}
\end{figure}

\subsection{Work Package 1: Storing and publishing personal genomic data in a decentralized environment} 

I will test the viability of Solid data pods as storage infrastructure for patient PGS data, thus, testing my hypothesis that Solid can support PGS data storage. 

A test dataset will be constructed using publicly available Illumina platinum genome files \cite{noauthor_platinum_nodate}. 
These files will be used as representative "patient" PGS data for experimentation. 

I will also create server-hosted Solid pods using the Community Solid Server (CSS) implementation of Solid \cite{css}. 
Each pod will be a storage container for a single individual\textquotesingle s PGS data. 
We will upload a single PGS file, a VCF file, into one "patient\textquotesingle s" pod to test basic functionality of a Solid pod for hosting large patient genomic data. 
All tasks within WP1, including test data set assembly, Solid pod creation and hosting using the CSS, and test data uploading to Solid pods will be evaluated only for functionality.

The use of the CSS for Solid pod hosting for research purposes is state-of-the-art, but the there have been no published experiments documenting the use of Solid pods for storing PGS data, which are much larger than in past Solid experimentation. 


\subsection{Work Package 2:  Storing PGS data in RDF and as Linked Data}

I hypothesize that the conversion from VCF to RDF is possible, and the resulting RDF representation will allow for linking of other medically relevant data within the patient\textquotesingle s pod and outside of it.
The conversion process will be made reversible to enable connection to existing clinical workflows that request VCF format. 

To convert PGS data from VCF to RDF, we will investigate a format translation process using the SPHN RDF ontology \cite{van_der_horst_bridging_2023}. 
During this translation process, we will experiment with different approaches, such as a bidirectional mapping index, for efficient reversal of conversion.
Because representation of VCF files in RDF has not been heavily studied, these will be the first experiments of their kind.

I then intend to demonstrate the linking of part of a patient\textquotesingle s genome to
(A) other data within the patient\textquotesingle s pod, 
(B) data in a public database outside of a patient\textquotesingle s pod, and
(C) data from another patient\textquotesingle s pod.

Direct conversion between VCF and RDF will be evaluated in terms of computational overhead, conversion time, and memory usage, both in the Solid pod and during conversion.
The same evaluations will also be performed on the process when an intermediate mapping index file is used. 
These comparisons will be documented in a formal benchmarking study.
Functionality of data linkage aims will be assessed by querying over the data in WP4.

While Linked Data is state-of-the-art, these concepts have not yet been applied to clinical genomic data.
The power of linking the VCF data to other clinically relevant data will be especially realized when these semantic links are discovered during querying, which will be investigated in WP5. 


\subsection{Work Package 3: PGS data privacy policies}

In this work package, I will experiment with the design and implementation of multiple levels of authorization as well as methods that allow for dynamic control over data discoverability, read/write access, and data access consent requests within a patient\textquotesingle s Solid pod. 
I hypothesize that various levels of authorization can be implemented and provide protections for maintaining the privacy of PGS data stored in Solid pods.

I will develop and test three functionalities for privacy modifications.
(1) registration of a pod to an individual patient,
(2) submission of a request to access stored data from a data requester, the notification of the patient, and the consent or denial by the patient, and
(3) permission revoking capabilities as well as an opt-in option to share their data with researchers. 
All of these methods will be integrated into the framework\textquotesingle s web application.
To utilize these methods, various levels of access to pod read and write privileges will be created to fill the needs and roles of participants of a PGS clinical workflow. 

Attaching differing levels of authorization to data will be assessed by creating various profiles that reflect clinical roles and access levels and attempting to access data via user-mediated, application requesting, and querying approaches. 

Assigning the above permissions within Solid is an open area of research and there are currently state-of-the-art protocols implemented in the CSS that allow their implementation.
The described access schema has not been attempted in the presented level of detail for clinical genomic data.


\subsection{Work Package 4: Querying over PGS data in one and many pods}

This work package will establish a querying mechanism for data in the patient Solid pods that takes into account patient pod data, user permissions, and data linkages. 
I hypothesize that a querying functionality that utilizes a query engine computational strategy will be able to query over patient Solid pods and return query results.

I will executing queries across PGS data contained in patient pod(s) through the use of the query language SPARQL \cite{noauthor_sparql_nodate}.
Query execution requires a source for computation which is not currently provided by the Solid pods themselves.
I will investigate the use of a query engine, such as that offered by Comunica \cite{comunica}, to perform the queries apart from the data stores.

For PGS data querying, I will benchmark and potentially build upon the link traversal query processing (LTQP) paradigm \cite{taelman_evaluation_2023}, which has been shown to be an effective method for querying within Solid.  
I will look to innovate and improve performance by combining existing algorithms with strategies that leverage the unique structure of PGS data such as the use of pre-computed indexes, like the one generated for RDF-VCF conversion, as a guide for faster query processing.

Query engine functionality will be evaluated using query execution time and computational load metrics as well as query results assessment. 
Query results will establish the functionality of data linkages from WP2.

Benchmarking will be done for existing LTQP algorithms and altered query algorithms that utilize genomic index files and results will be compared.
Ideally, success will be determined by queries that return correct results in under 10 minutes for users and potentially longer for applications.
In a clinical setting, time constraints are not as important as accuracy and reliability of results although excessive query times decrease the usefulness of such a tool for physicians in clinical practice.

LTQP algorithms are an active area of active research, but most of the work done has been with generalized algorithms and ,amy datastores with small amounts of data.
I aim to adapt this querying approach to the specific domain of genomic and health data which has not been attempted before. 


\subsection{Work Package 5: Component consolidation and framework deployment}

In an effort to improve data flows for research purposes, I intend to connect the proposed framework to the international Beacon initiative \cite{rambla_beacon_2022} to increase the availability of genomic data for researchers. 
In this aim we will investigate the necessary requirements and infrastructure necessary to connect patient Solid pods, containing PGS data, as beacon endpoints that can be discoverable and queried via the Beacon API. 
The connection of a decentralized, citizen-centric storage framework to the Beacon network is novel in nature as all other existing endpoints are institution-centric relational databases maintained by hospitals or research institutions.

All other functionalities will also be packaged into a web application with supporting documentation for final deployment and exhibition of how such a framework could function in clinical practice.
This framework would be the first of its kind.

Beacon API connection will be evaluated on functionality and integrate all previous work package components. 
Similarly, evaluation of the web application from which a user can interact with the framework will also be based on functionality.

\textbf{Planning:}
My project consists of 5 work packages denoted as “WP”. 
This will be bundled into a Ph.D. dissertation for a final thesis defense. Figure 2 shows a Gantt-chart of the Ph.D. project on a quarterly basis.  
Each work package is split up into different tasks with a dedicated amount of time allocated to it. 
This will allow for a good time and project management. 

[Figure 2: Insert gantt chart here]
[Figure 2 description]
\textbf{Figure 2 Gantt chart of the Ph.D. project timeline on a quarterly basis.} T denotes different tasks of a work package. M denotes different milestones, such as finishing and submitting a manuscript to a journal. 
Green denotes working on the thesis dissertation and defense. 
This is allocated as the final 6-9 months of the Ph.D. project to make all preparations for the defense. 


\section{4. Strategic dimension and application potential}
\begin{comment}
Elaborate the strategic dimension of your research, with regard to the (long-term) potential for innovative applications. 
Substantiate the PhD project’s strategic focus on economically relevant innovations. Justify how the chosen research approach (if successful) is the appropriate one to achieve the anticipated application(s) (potentially long term).
Elaborate the strategic importance of the potential applications to possible users (impact). Show how (if the project is successful) new products, services and/or processes may affect business of specific companies, a collective of companies and/or a sector and/or may be closely aligned with the Flemish science, technology and innovation transition priorities  (Flanders in transition. Priorities in Science, Technology and Innovation towards 2025) (socio-economic benefits). Societal impact should always be linked to a (in)direct (macro)economic benefit, e.g. cost reductions in health care, higher education level, environmental impact etc. should be positioned in an economic context.
\end{comment}

The infrastructure for sharing data between healthcare institutions is rapidly expanding but running into significant challenges such as lack of data interoperability, privacy and consent issues, as well as legal and regulatory restrictions. 
With the emergence of patient genomic data as a tool for clinicians, establishing the infrastructure for patient genomic data sharing is an economic niche that is largely unfilled. 
There certainly exists a fledgling private genomic service industry dominated by companies such as 23andMe, Ancestry.com, sequencing.com, and others establishing that genomics data generation and storage holds importance to consumers for various personal and medical reasons. 
At the same time, hospital systems exclusively store and maintain all patient PGS data that is used for clinical applications. 
There is notable nuance between these two sectors including different forms of genomic data being generated, stored, and used, differing legal oversight concerning commercial genomic data and health data, and formatting differences between the genomic data stored. 
Regardless, in our modern age of big data, data duplication due to data siloing, energy waste due to computational demands during data regeneration, and intrinsic security concerns for modern data storage techniques are major economic inefficiencies of the current system. 

A hypothetical company that, in coordination with policy makers and regulatory bodies, creates a scalable storage and data sharing infrastructure for genomic data, which could also grow to include all patient health data in time, stands to greatly increase the efficiency of PGS data usage in healthcare. 
Such efficiency increases could help lower patient costs for specialized genetic tests, remove data management and administration from hospitals, thereby reducing costs, and establish a new market within which economic growth could result. 

My project presented above is designed to present a proof-of-concept framework, both providing and demonstrating the technological foundations for the storage of PGS data in Solid pods, the controlling of access to that data on a granular level, the ability for that data to be queried, and exhibiting the accessibility of the stored PGS data to users, web applications, and medical tools in formats that can be used by both those currently in use and applications developed in the future. 
Such a framework will provide the outline of necessary implementation considerations from a technological perspective while also highlighting strengths and weaknesses of such a system that may be influential in attempts at scaling such an infrastructure. My project is also being undertaken parallelly with the Digital Twins of citizens/patients initiative that is happening at VITO in conjunction with the Flemish government and (?) for evolving the way medical data is stored to be increasingly citizen/patient centric. 
This initiative along with the WE ARE project at VITO are both exploring the ways in which decentralized storage could be applied to sensitive data to improve the way that consent is given and requested for such data. 

Notably, the framework I am developing is intended to augment and contribute to advances in medical patient care by removing existing cost and architectural barriers to using PGS data more broadly in clinical practice. 
In the short-term, the project is being developed to be integrated into ongoing research and product development at VITO Digital Precision Health. 
Products aimed at improving the way drug prescription is practiced by using a genetic screening tool that leverages documented genetic predispositions to drug ineffectiveness are currently being developed to be connected to my framework of Solid pod stored PGS data. 
Additionally, connection to other known and well-used workflows such as for NIPT and rare genetic disease screening is a primary goal for my project. 
Genomic data interoperability is of utmost importance for clinical application and is therefore a cornerstone of my project. 

Lastly, public perception is a crucial element to the economic growth of a product or sector. With personal data usage transparency as well as greater calls for digital data privacy protections becoming more important to the public, such considerations should also be priorities to how health data is managed. 
The existing system of genomic data storage for use in healthcare is prone to data leaks and heavily restricted patient transparency due to the central architecture of institution-centric data stores. 
With my proposed framework, patients would be more intimately connected to their data, potentially even having a say over to whom and what their data is visible. 
Such improved transparency, when paired with decreased risk of large-scale data leaks, is likely to be well-received by the general public. 
Such public support could help drive such a framework adoption to a larger scale such as nationally or even to be the standard for a system like the EU. 
This large scale goal, while nowhere near attainable in the near future, would present the greatest possible outcome for such a project and exhibit a somewhat unintuitive increase in greater genomic data privacy and shareability. 
In this scenario, there is also room for healthy competition within such a niche as various pod providers could offer hospital systems and educational institutions different rates for data storage and associated computation.

*** Disruptive Innovation --> Economic added value ***
	--Especially specifically for Flanders–

\bibliography{FWO_Project_Description_EDC}
\end{document}