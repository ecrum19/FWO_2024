
\section{Abstract}
    Medical care is in the process of becoming increasingly personalized through the use of patient genetic information. 
    At present, data useful for clinical care, including genetic data, is commonly diffuse, organized arbitrarily, and stored in data silos. 
    Thus, unstructured organization, high costs for data storage and generation, and tight privacy restrictions pose serious challenges to scaling personalized clinical strategies.
    I propose an early stage Ph.D. that aims to improve the connectedness and shareability of genomic data storage(s), while preserving data privacy, to decrease the costs of using patient genome data in clinical practice. 
    In this pursuit, I will integrate various domains of semantic web research into a novel, holistic framework designed for use in clinical practice.
    Specifically, I will (a) store patient data using Solid pods, (b) represent personal genome sequence data in RDF as Linked Data, (c) attach policies to stored data, and (d) query data through link traversal queries.

\keywords{Solid \and Linked Data \and Querying \and Genomic Data Sharing}


\section{Position your proposal in terms of economic finality
\textit{
Ultimately (medium- to long-term), the proposed strategic research may lead to added value for one or more specific company(ies), or for a sector or group of enterprises. 
The application potential may as well be expressed in terms of socio-economic benefits, related to the Flemish transition areas and priorities in science, technology and innovation. 
You can highlight multiple options simultaneously but you need to select at least one. 
In the first two cases you have to specify which companies or sectors are targeted. Furthermore, you can select up to 2 transition areas, each with an associated priority. It is also possible to choose two priorities under the same transition area. What you mention in this section should be referred to, elaborated and explained in the Project description, section ‘strategic dimension’: Hence, do not just drop company names here, and be specific in referring to sectors (be more precise than e.g. ‘manufacturing’ or ‘space industry’).
}

\textbf{Companies (optional)}
My project is nested with the larger goals of VITO NV's Digital Precision Health group to develop tools and technologies that encourage increasingly citizen-centric health care.

\textbf{Sectors (optional)}
Transition area group - Health & Well-Being
Transition area - Central electronic health record

Transition area group - Digital Society
Transition area - Next generation networks

My project targets the interation between heath data management, an individual's electronic health record, and the growing network of clinical and commercial applications for which these data will act as inputs.


\section{Explain any career breaks.}
\textit{
Explain possible gaps in your CV in the input field below. 
Make sure your current position and previous appointments are well listed in the e-portal ‘Personal details’ section (’Posts / Career’). 
If you have interrupted your academic career at any given point for at least three months (maternity leave, parental leave, full-time sickness leave, unconventional career paths such as leave because of activities in industry or other non-academic sectors, …) provide details about this below (reason, start/end date). This will allow the reviewers to fairly assess your career stage.
}

My career as a student and researcher was briefly interrupted after the completion of my Master's program in fall 2022, before I began my Ph.D. at UGent in fall 2023. 
During this year I was not formally enrolled in an academic program, but I was not really away from academia. 
During this time I was employed as an adjunct professor of biology at Loyola University Chicago in Chicago, USA.
For a half of that year, I was also employed part time as a medical scribe in the emergency room of a university hospital in Chicago.
In this year I was also finishing off some research project loose ends that were not fully completed for my Master's thesis defence, including a primary author peer-reviewed journal publication.
While I was sufficiently busy during my year off, I also took significant time to reflect and subsequently made some influential decisions about my future.
The most crucial being my decision to pursue a Ph.D. instead of medical school as well as my decision that my Ph.D. candidacy would be in Flanders at Ghent University.


\section{Study narrative.}
\textit{
Show how your academic study trajectory has formed the ideal preparation for doing a PhD, in general and specifically on the topic of the proposed project. 
Where appropriate, refer to your grades of relevant courses, percentiles or relative ranking or other study results. 
You may also highlight specific programs or courses you took. 
If applicable, include additional information on your personal situation where you believe this may have affected your study results and should need to be taken into consideration during the evaluation.
}

My academic career has been both highly challenging and rewarding. 
As is exemplified by my bachelor's and master's academic record, as well as my graduation with honors from my Master's program, I am a highly capable student, adept at excelling in an academically challenging environment, and keen to learn.
The courses that I have followed over my journey toward a Ph.D. have taught me skills, knowledge, and strategies to succeed at such an endeavor.
Specifically, within biology, I have excelled in Genomics, Molecular Biology, and Proteomics courses, receiving a perfect score in all of them, which has provided me with the conceptual and scientific foundations for pursuing the genomic components of my Ph.D. 
Within bioinformatics, I followed courses such as computation biology, quantitative bioinformatics, and bioinformatics, completing all with a perfect score, where I learned about the computational, statistical, and data science aspects of quantitative biological research. 
For foundational skills and knowledge in computer science, classes such as data structures, discrete structures, computational biology, and object-oriented programming, taught me about the architecture, concepts, and strategies leveraged in computational science research.
These courses also taught me coding proficiency in Python, Java, SQL, Bash, as well as other crucial software development fundamentals. 
In my master's coursework, I followed classes such as responsible conduct for research, bioinformatics research design, and scientific writing which helped prepare me for the intangible aspects of Ph.D. research crucial to researcher success.
For my proposed Ph.D., my academic excellence in biology/bioinformatics, computer science, and research logistics courses has well prepared me for success while also providing me with experience and confidence that I can problem solve, learn challenging new knowledge domains, and communicate my findings with fellow researchers and scientific lay people alike.



\section{Text field to provide additional information on your study results (the global percentage, percentile, rank in study group). (optional)}
\textit{
If you were not able to provide a global percentage and/or positioning in the study group, you can use this text field to present in a qualitative way the relative positioning of all your study results compared to your peers. You can also use this text field to provide additional information on your academic study results (Bachelor, Master, Advanced Master, ...), i.e. detailed course scores can be added or, if you have not yet obtained your master, you can marks obtained in the first master year All evidence on study results should be uploaded in 'Personal Details/studies' section.
}

My positioning in my Masters class is reflected above.
My Bachelor's graduating class did not record class rankings but there were 2,934 bachelor's level graduates and I recieved a 3.991/4.0 cumulative bachelor's GPA. Thus, I finished with 99.775% in my bachelor's program. For comparability, a minimum of a 2.0 GPA is needed to graduate and is considered a passing grade.
Also to note, I was admitted to the Alpha Sigma Nu honors society, from which only the top 10% of each academic class is invited to apply.


\section{Write a motivation statement.}
\textit{
Elaborate on your motivation and research interests to pursue an individual PhD trajectory. 
Elaborate also on how your scientific background and competences will allow you to start the PhD project, and to grow into a strategically thinking and innovation- oriented expert. 
Provide a clear and substantiated overview on the skills you have already developed, and on the competences yet to be acquired and how you will acquire them.
}

Curiosity is a cornerstone of my identity. 
As a student, I quickly fell in love with learning as a way to exercise my inate curiosity. 
Entering my bachelor studies, I was determined to become a physician because of my love of biology and personal history with serious health issues as a child. 
As a bachelor's student, I chose to pursue an interdisciplinary field that moved me towards my goals of becoming a physician while also allowing me to learn about a field I knew nothing about -- computer science.
Quickly I fell in love with genomics, specifically the complexity presented by genomics data, and the computational heuristics developed to learn about and leverage such data.
Gradually, I shifted my goals from medicine to pursuing computational medically motivated research.
During my Master's thesis project, I developed a strong foundation of research skills and genomics domain knowledge.
Specifically, I because well acquianted with digital human genomics data formats and uses, compression, indexing, and parsing strategies for these complex and large data, and computational competancies in data processing, coding, and algorithm application.

Ultimately, my curiosity about data representation and data parsing optimization that began during my master's project drove me to begin my Ph.D. at Ghent University studying how representing genomic data using semantic web and decentralized storage technologies could improve clinical data flows.
I am new to the field of semantic web research, as well as the many sub-domians that focus on aspects such as data policies and governance, semantic representations of data, and Web technologies.

I also bring with me expertise in genomics data structure, parsing strategies, and algorithmic approaches to biological problem solving. 
As is exemplified by my graduation with honors from my Master's program, I am a highly capable student, adept at excelling in an intellectually challenging research environment, and keen to learn.
I will fill the gaps of my current lack of formal knowledge of semantic web technologies and concepts through mentoring and interactions with members of my research group, the Knowledge on the Web Scale (KNoWS) group within IDLab, at Ghent University, as this is a domain of research at which they excel.
I am also applying to attend a semantic web summer school this coming summer, looking to attend conference tutorials, as well as potentially following select masters level university courses that can provide me specific domain-specific knowledge.


%%
1. I am passionate about genomics and algorithmic design. 
In my past research, I used BLAST+ extensively and have always been fascinated by how heuristic algorithms like seed-and-extend can return accurate, rapid results when searching large query sequences against gigantic subject databases. 
I will admit that I do not have extensive experience with the design or improvement of such algorithms, but I have always had an interest in optimization and solving complex NP-Hard coding problems in clever ways. 
I have 5+ years of experience coding in python and R, as well as experience with JavaScript, SQL, and BASH scripting. 
My other applicable experience comes from data processing and utilizing existing biological tools for genomic analysis which was a large part of my Master’s thesis. 
In the next stage of my career, I am extremely interested in pursuing the development of new algorithms and novel application of existing querying algorithms in a complex data structure environment, like is presented by the Solid datastore ecosystem.

2. This project presents the type of structured, difficult computational challenge that I am looking for in a PhD project. 
While challenging and PhD projects are synonymous, I am attracted to the Query Execution over Personal Genomes project because of the way in which it is challenging. 
The project will doubtless be quite an endeavor, but at the outset, the deliverable is known and achievable. 
Within this problem framework, I am excited about the prospect of studying the cutting edge of database communication, complex Solid ecosystem genome querying strategies, existing algorithms designed for data searching in the Solid system, and the creation of a UI attractive and intuitive enough for public use. 
Overall, this project sounds like something that I am quite excited to lose myself in for the next four years of my life. 

3. The project location fits with my non-academic aspirations. 
Through experiences studying abroad and traveling, I have determined that I want to pursue a PhD in Europe. 
I loved my education in the US, but I am adventurous and determined to spend the next chapter of my life in Europe. 
The marriage of this PhD project’s location in Belgium and being a project that aligns with my experience and future goals makes it a once-in-a-lifetime opportunity.

4. The deliverable product will have an impact on society and prepare me for a career in an increasingly impactful field. 
This PhD aligns with my aims to use my expertise in genomics and computer science to improve the lives of others. 
Producing a platform created to have a wide-spread impact through improvements in human genome accessibility for researchers, increased genomic privacy for users, and accessibility of information for clinicians is a project that I would be proud to pursue.

My passion for learning and pursuing difficult computational problems as well as my dedication to work hard make me well suited for the position. 
I am extraordinarily excited about the opportunity to join the team and look forward to taking the next step on this journey.


\section{Scientific activities, experiences and achievements.}
\textit{
In this input field you can further elaborate on first steps as a (potential) innovation-oriented scientist. List relevant activities, experiences and achievements that may be relevant for assessing your potential to start a PhD. For mobility and awards, other dedicated input fields are available below.

For (ongoing or finished) master thesis or equivalent (as well as dissertation advanced master): mention title, promotor, research group and host institution. If the thesis is related to your PhD topic, also specify initial objective, methodology used and (intermediate) results.

For (PhD) research already started, specify initial objective, methodology used and (intermediate) results.
If applicable mention (up to 5) publications and other achievements. Mind, do mention for each achievement item (publications and other achievements) your share and its nature, and those of other significant partners in the workload.

For publications: list all authors, title of publication and journal name (without abbreviations) with volume, start/end page and year. Mention whether the publication was peer reviewed or not. For book publications, give all necessary bibliographic information (author(s) or editor(s), book title, publisher, place, year, number of pages).
Make sure your complete publication list is up to date in the e-portal ‘Personal details’ section (“Publications”).
For other achievements: provide a short description, when it was undertaken and finalised and list all the relevant participants involved in it.

List any other distinct research output that does not fit in the bibliographic publication list and that is meaningful in a broad sense with respect to this fellowship application. It may be constituted by a data base, surveys, a technical diagram, software, objects (maquettes, prototypes…), any other type of activity or output you consider to be relevant. Date the output where appropriate.

Mention any relevant, past or concretely planned, experiences (internships, presentations, collaborations, …)
}

I have successfully defended a master's thesis in Bioinformatics. Project Title: CATALOGING AND ENGINEERING TEMPERATE COLIPHAGES OF THE HUMAN
URINARY MICROBIOME; Promoter: Dr. Catherine Putonti; Putonti Lab research group; Loyola University Chicago. 

During the first months of my Ph.D., I have been composing a scoping review paper on the current landscape of clinical genomic data sharing.
I plan to submit the review paper for publication in a peer-reviewed journal in the coming months.
I have presented a poster at the Semantic Web Applications and Tools 4 Health Care and Life Sciences 2024 conference that proposes the use of Solid data vaults for storing genomic and health data.
I composed the accepted 2-page poster abstract, produced the physical poster, and presented the poster at the conference.
I was the contributing author for a consortium poster also presented at the Semantic Web Applications and Tools 4 Health Care and Life Sciences 2024 conference. I helped provide input for the 2-page poster abstract, helped contribute to the composition of the physical poster, and was one of the consortium members that presented the poster at the conference.
Within WP1, I have successfully assembled the test dataset and set up CSS pod instances.
I have also successfully upload VCF files into these pods signalling completion of WP1. 

I was the primary author of the publication titled: Coliphages of the human urinary microbiota
Authors: Elias Crum, Zubia Merchant, Adriana Ene, Taylor Miller-Ensminger, Genevieve Johnson, Alan J Wolfe, Catherine Putonti
Journal: Plos one; Volume 18; Issue 4; Pages e0283930
Peer Reviewed: yes
Date Published: 2023/4/13

I was the contributing author of the publication titled: Genome Investigation of Urinary Gardnerella Strains and Their Relationship to Isolates of the Vaginal Microbiota.
Authors: Catherine Putonti, Krystal Thomas-White, Elias Crum, Evann E Hilt, Travis K Price, Alan J Wolfe
Journal: Msphere; Volume 6; Issue 3; Pages e00154-21
Peer Reviewed: yes
Date Published: 2021/6/30

I have presented two posters at conferences during my research stay at Loyola University Chicago, both at the St. Albert Day intra-university research symposium held annually for biological research performed by undergraduate, graduate, and medical students.
These poster were not published due to the inter-university aspect of research.




\section{Specify earlier mobility (research stays) in other organizations.}
\textit{
Indicate the research stays which have already been undertaken, prior to this project. 
If applicable, motivate shortly the added value of each stay to the project. 
Include details on the organization, type of organization, country, contact person, start/end date, function/activities.
}

Research stay 1: Stritch School of Medicine, Wolfe Lab, Summer 2019
Country: USA; Contact Person: Dr. Michael J. Wolfe; Start May 2019 / End August 2019; Research assistant

This project was both microbiological and computational in nature.
I became familiar with the methods, practices, and procedures of scientific researchers as well as distinct data collection and reporting techniques. 
I learned how to communicate my work with colleagues, extract meaning from data, and engage with colleagues to problem solve and provide constructive criticism in weekly meetings with researchers and clinicians.
I was challenged by learning to fully understand the complex scientific underpinnings of my project and then by communicating those concepts effectively to diverse audiences. 
For a final presentation, I gave a 30 minute presentation to around 80 principal investigators, post-doctoral researchers, and graduate students. 

Research stay 2: Loyola University Chicago, Putonti Lab, Fall 2019-Spring 2023
Country: USA; Contact Person: Dr. Catherine Putonti; Start Sept 2019 / End April 2023; Independent researcher.

As a student researcher in Dr. Putonti’s lab at Loyola, I perform independent work focusing on bacteriophage and bacteria of the urinary microbiome at both bachelor and master levels. 
My research involved both computational analysis of bacterial and viral genomes as well as molecular biologically-based lab techniques. 
Specifically, I worked to determine the genomic contributions to a host range shift in the 3 bacteriophage, worked to characterized bacterial and viral phylogenic diversity based on genomic and genetic similarity analyses, and engineered bacteriophages to selectively kill specific strains of bacteria. 
Actively performing research has taught me that very few experiments will work as planned -- problem-solving, flexibility, and persistence are prerequisites to discovery and success, and when planning, organization and details are crucial. 





\section{Specify concrete mobility plans (research stays) within the FWO fellowship.}
\textit{
Indicate the research stays which are planned within the FWO fellowship. 
Motivate shortly the added value of each stay for the project. Include details on the organization, type of organization, country, contact person, start/end date, function/activities. 
}

I do not currently plan to pursue any research stays at external institutions during the duration of my Ph.D.


\section{List any scientific awards.}
\textit{
List prizes and awards, (e.g. best master thesis…). Specify the awarding body, title, date, amount and theme.
}

I was awarded a Mulcahy research scholarship by Loyola University Chicago for the 2020-2021 school year in November 2021 which was for the amount of $3000. 

I was awarded an institutional scholarship by Loyola University Chicago for my Master's program (2021-2022 academic year) for a total of $20000 in spring 2020.

I was inducted to the Jesuit honors society Alpha Sigma Nu in October 2019 based on distinction in scholarship, service, and faith.



\section{Explain how this project fits into the research activities of the involved host institution(s).}
\textit{
Elaborate on the positioning and embedding of your project in the research group(s), its scientific as well as strategic ambitions. 
If applicable, also position your own previous and current research to the proposed PhD fellowship project.
}

The IDLab, and specifically the KNoWS group at Ghent University, works in multiple semantic web research domains that can be integrated into decentralized storage technologies such as Solid.
These fields include data privacy and governance, decentralized federated querying, and big data semantic representations.
My project touches on each of these domains and aims to integrate state-of-the-art solutions from each into a collective framework.
Further, my Ph.D. explores situations currently unexplored but promising for future research within our group, especially related to large data storage in Solid pods and querying of that data.
Within UGent's SolidLab project, my Ph.D. will also provide a practical implementation of Solid technology to a biomedical knowledge domain that allows for novel boundary exploration.

The Digital Precision Health group at VITO aims to develop and implement products that improve the scalability of personalized medicine.
My project is embedded in this mission by aiming to demonstrate how a decentralized storage structure on which other applications, tools, and software can be built and integrated. 
In this aim, I will also combine state-of-the-art semantic web technologies into a user-friendly application that allows for non-experts, like physicians or patients, to utilize the advantages of these technologies.
In this way, my project will demonstrate a path toward product production that encourages future clinical implementation.



\section{Position the project in a national and international context.}
\textit{
Mention specific research collaborations planned in the course of this project, if appropriate, mention larger projects, programmes or networks your proposal may be part of.
}

Solid data pods offer discoverable, privacy controlled interoperable storage for RDF data. 
Because of this property, a collaborative project with the Swiss Institute of Bioinformatics and Institute of Organic Chemistry and Biochemistry of the Czech Academy of Sciences has been created to explore their potential as data stores for sensitive research data that can be queried in a federated manner along with existing large, public SPARQL endpoints to answer research questions.
This consortium has been awarded a CHIST-ERA Call ORD research grant for the years of 2023-2025.

My project is also designed to fit within the European Virtual Human Twin (EDITH) initiative. 
I will be working to design a platform that could be integrated within and/or provide inspiration for storage and sharing strategies within the EDITH initiative.

My project, in representing genomic data as RDF, is also dedicated to encouraging greater data interoperability embodied by the European FAIR data initiative.

I aim to integrate my framework to connect to or align with Belgian Genome Biobank project and system architecture that is planned to be implemented in 2026.



\section{Indicate how the results of the proposed research will be communicated to a non-expert audience.}
\textit{
FWO encourages its fellows to disseminate the results of their research widely and valorise them where possible.
}

My project is designed to produce a proof-of-concept framework that can be used by non-experts. 
This may be the most attractive aspect of this project, in that it is designed to be for non-expert audiences.
Thus, my largest communication goal will be implementation of the framework, making that implementation findable and usable over normal internet browsers, and demonstration of how the web application can be used.
This will hopefully be possible for both those within and outside of the scientific community through internet availability. 
It should be said that all usable implementations of my project will only contain publicly available genome data, example patient data that only representational, and no way to actually input or discover real personal patient data.

I will also host a personal website that communicates intermediate results, methodologies, and specific implementations. 
I plan to do this through regular blog posts, video and written tutorials, and github-hosted software advertisements. 
Interaction with more general lay audiences will be attempted through X and Mastadon.

I will devote the most time and effort to blog posts and their advertisement.


\section{Motivate your choice of expert panel.}
\textit{
Carefully read the scientific scope of the selected expert panel and motivate why your application fits the scope of this panel - 
i.e. why this panel has the most appropriate expertise to evaluate your proposal.
}

My project can be described as the unification of cutting-edge semantic web technologies into a functional framework that provides the digital infrastructure for improvement of data flows in a biomedical knowledge domain.
Thus, my project can be described as applied web and information system engineering covered by the SBWT5B - Informatics and data communication panel.
This project is strategic in nature because of its goal to improve the state-of-the-art in clinical genomics data handling. 



\section{Describe the datatypes (surveys, sequences, manuscripts, objects …) you will collect and/or generate and/or (re)use during your research project.}
\

-Genome Data: VCF formatted whole genome sequence data downloaded from public genome repositories.
- Research articles: Final PDF, HTML, and LaTeX representations of papers submitted to conferences and journals, including all sources for figures.
- Software: All implementations and documentation of Solid storage, algorithms, techniques, and supporting tools for running experiments.
- Benchmarks: Queries and datasets for running experiments, with detailed documentation.
- Experimental results: Raw output from benchmarks in machine-readable formats from all used metrics.
- Presentations: Slides and sources for compiling them. If applicable, recordings of the presentation.



\section{Specify in which way the following provisions are in place in order to preserve the data during and at least 5 years after the end of the research.}

IDLab has a strong archiving environment, which is maintained by our IT support team, led by Brecht Vermeulen. 
This enables preservation of all my research outcomes for at least five years thanks to nightly redundant backups.
At VITO, similar archiving with multiple backups of the cloud environment on which all files are stored is done. 
To increase redundancy, I store all my research data in git repositories linked to my personal account.


\section{Which other issues related to the data management are relevant to mention?}

Not sure? Nothing I can think of?
