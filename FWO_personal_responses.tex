
\section{Abstract}
    Medical care is in the process of becoming increasingly personalized through the use of patient genetic information. 
    At present, data useful for clinical care, including genetic data, is commonly diffuse, organized arbitrarily, and stored in data silos. 
    Thus, unstructured organization, high costs for data storage and generation, and tight privacy restrictions pose serious challenges to scaling personalized clinical strategies.
    I propose an early stage Ph.D. that aims to improve the connectedness and shareability of genomic data storage(s), while preserving data privacy, to decrease the costs of using patient genome data in clinical practice. 
    In this pursuit, I will integrate various domains of semantic web research into a novel, holistic framework designed for use in clinical practice.
    Specifically, I will (a) store patient data using Solid pods, (b) represent personal genome sequence data in RDF as Linked Data, (c) attach policies to stored data, and (d) query data through link traversal queries.

\keywords{Solid \and Linked Data \and Querying \and Genomic Data Sharing}



\section{Position your proposal in terms of economic finality.}
\textit{
Ultimately (medium- to long-term), the proposed strategic research may lead to added value for one or more specific company(ies), or for a sector or group of enterprises. 
The application potential may as well be expressed in terms of socio-economic benefits, related to the Flemish transition areas and priorities in science, technology and innovation. 
You can highlight multiple options simultaneously but you need to select at least one. 
In the first two cases you have to specify which companies or sectors are targeted. Furthermore, you can select up to 2 transition areas, each with an associated priority. It is also possible to choose two priorities under the same transition area. What you mention in this section should be referred to, elaborated and explained in the Project description, section ‘strategic dimension’: Hence, do not just drop company names here, and be specific in referring to sectors (be more precise than e.g. ‘manufacturing’ or ‘space industry’).
}

\textbf{Companies (optional)}
My project is nested with the larger goals of VITO NV's Digital Precision Health group to develop tools and technologies that encourage increasingly citizen-centric health care.

\textbf{Sectors (optional)}
Transition area group - Health & Well-Being
Transition area - Central electronic health record

Transition area group - Digital Society
Transition area - Next generation networks

My project targets the interation between heath data management, an individual's electronic health record, and the growing network of clinical and commercial applications for which these data will act as inputs.


\section{Explain any career breaks.}
\textit{
Explain possible gaps in your CV in the input field below. 
Make sure your current position and previous appointments are well listed in the e-portal ‘Personal details’ section (’Posts / Career’). 
If you have interrupted your academic career at any given point for at least three months (maternity leave, parental leave, full-time sickness leave, unconventional career paths such as leave because of activities in industry or other non-academic sectors, …) provide details about this below (reason, start/end date). This will allow the reviewers to fairly assess your career stage.
}

I took one academic year off after the completion of my Master's program in fall 2022, before I began my Ph.D. at UGent in fall 2023. 
During this year away from academia, I was not really away from academia because I was employed as an adjunct professor of Biology at Loyola University Chicago in Chicago, USA.
For a quarter of that year, I was also employed part time as a medical scribe in the emergency room of a hosptial in Chicago.
In this year I was also finishing off some research project loose ends that were not fully completed for my Master's thesis defeanse.
While I was sufficiently busy during my year off, I also took significant time to reflect and subsequently made some influential decisions about my future.

\section{Study narrative.}
\textit{
Show how your academic study trajectory has formed the ideal preparation for doing a PhD, in general and specifically on the topic of the proposed project. Where appropriate, refer to your grades of relevant courses, percentiles or relative ranking or other study results. You may also highlight specific programs or courses you took. If applicable, include additional information on your personal situation where you believe this may have affected your study results and should need to be taken into consideration during the evaluation.
}



\section{Write a motivation statement.}
\textit{
Elaborate on your motivation and research interests to pursue an individual PhD trajectory. 
Elaborate also on how your scientific background and competences will allow you to start the PhD project, and to grow into a strategically thinking and innovation- oriented expert. 
Provide a clear and substantiated overview on the skills you have already developed, and on the competences yet to be acquired and how you will acquire them.
}

As an aspiring computational genomics innovator and recent graduate with a Master’s degree in Bioinformatics, it is with great excitement and hope that I am applying for the Query Execution over Personal Genomes PhD position. 
As a student, I pursued an interdisciplinary field that integrates two of my passions – biology and computer science – with the hopes of discovering and/or creating things that improve the lives of others. 
During my bioinformatics studies, I became further fascinated with genomics and the societal impacts that could be drawn from improving our understanding of the human genome. 
Since my graduation in August, I have been working to decide in which direction I can best pursue my passions and contribute to improving the world around me. 
It is with these motivations that I am applying to the Query Execution over Personal Genomes PhD position. 
In detail, there are four main reasons this project is right for me, and I am right for this project, which are detailed below.

1. I am passionate about genomics and algorithmic design. 
In my past research, I used BLAST+ extensively and have always been fascinated by how heuristic algorithms like seed-and-extend can return accurate, rapid results when searching large query sequences against gigantic subject databases. 
I will admit that I do not have extensive experience with the design or improvement of such algorithms, but I have always had an interest in optimization and solving complex NP-Hard coding problems in clever ways. 
I have 5+ years of experience coding in python and R, as well as experience with JavaScript, SQL, and BASH scripting. 
My other applicable experience comes from data processing and utilizing existing biological tools for genomic analysis which was a large part of my Master’s thesis. 
In the next stage of my career, I am extremely interested in pursuing the development of new algorithms and novel application of existing querying algorithms in a complex data structure environment, like is presented by the Solid datastore ecosystem.

2. This project presents the type of structured, difficult computational challenge that I am looking for in a PhD project. 
While challenging and PhD projects are synonymous, I am attracted to the Query Execution over Personal Genomes project because of the way in which it is challenging. 
The project will doubtless be quite an endeavor, but at the outset, the deliverable is known and achievable. 
Within this problem framework, I am excited about the prospect of studying the cutting edge of database communication, complex Solid ecosystem genome querying strategies, existing algorithms designed for data searching in the Solid system, and the creation of a UI attractive and intuitive enough for public use. 
Overall, this project sounds like something that I am quite excited to lose myself in for the next four years of my life. 

3. The project location fits with my non-academic aspirations. 
Through experiences studying abroad and traveling, I have determined that I want to pursue a PhD in Europe. 
I loved my education in the US, but I am adventurous and determined to spend the next chapter of my life in Europe. 
The marriage of this PhD project’s location in Belgium and being a project that aligns with my experience and future goals makes it a once-in-a-lifetime opportunity.

4. The deliverable product will have an impact on society and prepare me for a career in an increasingly impactful field. 
This PhD aligns with my aims to use my expertise in genomics and computer science to improve the lives of others. 
Producing a platform created to have a wide-spread impact through improvements in human genome accessibility for researchers, increased genomic privacy for users, and accessibility of information for clinicians is a project that I would be proud to pursue.

My passion for learning and pursuing difficult computational problems as well as my dedication to work hard make me well suited for the position. 
I am extraordinarily excited about the opportunity to join the team and look forward to taking the next step on this journey.


\section{Scientific activities, experiences and achievements.}
\textit{
In this input field you can further elaborate on first steps as a (potential) innovation-oriented scientist. List relevant activities, experiences and achievements that may be relevant for assessing your potential to start a PhD. For mobility and awards, other dedicated input fields are available below.

For (ongoing or finished) master thesis or equivalent (as well as dissertation advanced master): mention title, promotor, research group and host institution. If the thesis is related to your PhD topic, also specify initial objective, methodology used and (intermediate) results.

For (PhD) research already started, specify initial objective, methodology used and (intermediate) results.
If applicable mention (up to 5) publications and other achievements. Mind, do mention for each achievement item (publications and other achievements) your share and its nature, and those of other significant partners in the workload.

For publications: list all authors, title of publication and journal name (without abbreviations) with volume, start/end page and year. Mention whether the publication was peer reviewed or not. For book publications, give all necessary bibliographic information (author(s) or editor(s), book title, publisher, place, year, number of pages).
Make sure your complete publication list is up to date in the e-portal ‘Personal details’ section (“Publications”).
For other achievements: provide a short description, when it was undertaken and finalised and list all the relevant participants involved in it.

List any other distinct research output that does not fit in the bibliographic publication list and that is meaningful in a broad sense with respect to this fellowship application. It may be constituted by a data base, surveys, a technical diagram, software, objects (maquettes, prototypes…), any other type of activity or output you consider to be relevant. Date the output where appropriate.

Mention any relevant, past or concretely planned, experiences (internships, presentations, collaborations, …)
}

I have successfully defended a Master's Thesis in Bioinformatics. Project Title: CATALOGING AND ENGINEERING TEMPERATE COLIPHAGES OF THE HUMAN
URINARY MICROBIOME; Promoter: Dr. Catherine Putonti; Putonti Lab research group; Loyola University Chciago. 


During the first months of my Ph.D., I have been composing a scoping review paper on the current landscape of clinical genomic data sharing.
I plan to submit the review paper for publication in a peer-reviewed journal in the coming months.

I have presented a poster at the Semantic Web Applications and Tools 4 Health Care and Life Sciences 2024 conference that proposes the use of Solid data vaults for storing genomic and health data.
I composed the accepted 2-page poster abstract, produced the physical poster, and presented the poster at the conference.

I was the contributing author for a consortium poster also presented at the Semantic Web Applications and Tools 4 Health Care and Life Sciences 2024 conference. I helped provide input for the 2-page poster abstract, helped contribute to the composition of the physical poster, and was one of the consortium members that presented the poster at the conference.

Within WP1, I have successfully assembled the test dataset and set up CSS pod instances.
I have also successfully upload VCF files into these pods signalling completion of WP1. 


I was the primary author of the publication titled: Coliphages of the human urinary microbiota
Authors: Elias Crum, Zubia Merchant, Adriana Ene, Taylor Miller-Ensminger, Genevieve Johnson, Alan J Wolfe, Catherine Putonti
Journal: Plos one; Volume 18; Issue 4; Pages e0283930
Peer Reviewed: yes
Date Published: 2023/4/13

I was the contributing author of the publication titled: Genome Investigation of Urinary Gardnerella Strains and Their Relationship to Isolates of the Vaginal Microbiota.
Authors: Catherine Putonti, Krystal Thomas-White, Elias Crum, Evann E Hilt, Travis K Price, Alan J Wolfe
Journal: Msphere; Volume 6; Issue 3; Pages e00154-21
Peer Reviewed: yes
Date Published: 2021/6/30


\section{Specify earlier mobility (research stays) in other organizations.}
\textit{
Indicate the research stays which have already been undertaken, prior to this project. 
If applicable, motivate shortly the added value of each stay to the project. 
Include details on the organization, type of organization, country, contact person, start/end date, function/activities.
}

Research stay 1: Stritch School of Medicine, Wolfe Lab, Summer 2019
Country: USA; Contact Person: Dr. Michael J. Wolfe; Start May 2019 / End August 2019; Research assistant

I worked on my first independent research project specifically investigating the variable infectivity of seven bacteriophages among 68
strains of E. coli isolated from human bladders. My schedule in Dr. Wolfe’s lab included weekdays, nights and weekends to ensure proper
and timely advancement of our research goals. I participated in weekly meetings with researchers and clinicians to discuss the
significance of recent findings and adjust future aims. I became familiar with the methods, practices, and procedures of scientific
researchers as well as distinct data collection techniques. I presented the findings to the Stritch School of Medicine
Microbiology/Immunology Department.
This experience taught me not only how clinically-based research is performed but importantly it exposed me to physicians and
researchers working together to improve patient outcomes in disease treatment and prevention. The research I performed at Stritch was
my first experience with scientific research and provided me with a unique learning experience after my sophomore year in college. I
learned first-hand that medical research is extremely time intensive and requires a high level of commitment. I learned how labs operate,
how research is planned, as well as the many logistical nuances that are essential to obtaining informative data. I also developed a
greater respect for how small details impact the big picture within biological systems which led me to think in a more logical, inquisitive
way. I was challenged by learning to fully understand the complex scientific underpinnings of my project and then by communicating
those concepts effectively to a diverse audience. For my final presentation, I gave a 30 minute presentation to around 80 principal
investigators, post-doctoral researchers, and graduate students. I participated in weekly meetings with a number of physicians and
researchers and developed skills for communicating complex concepts in a simple and effective manner.


Research stay 2: Loyola University Chicago, Putonti Lab, Fall 2019-Spring 2023
Country: USA; Contact Person: Dr. Catherine Putonti; Start Sept 2019 / End April 2023; Independent researcher.

As a student researcher in Dr. Putonti’s lab at Loyola, I perform independent work focusing on bacteriophage and bacteria of the urinary
microbiome. My research involves both computational analysis of bacterial and viral genomes as well as molecular biologically-based lab
techniques. Specifically, I am working to determine the genomic contributions to a host range shift in T3 bacteriophage. As part of my
responsibilities, I help maintain the lab by cleaning, organizing, and restocking necessary materials. I was awarded a research
scholarship for the 2020-2021 school year which will culminate in presenting a poster to researchers and medical professionals around
April 2021.
I have worked on a wide range of research projects, of both my own design and the projects of others. Most work relates to the human
urinary microbiome (UMB), particularly UMB bacteriophages, over the past 4 years in the Wolfe Lab at Loyola’s Stritch School of
Medicine (Summer 2019) and in the Putonti Lab at Loyola’s Lake Shore campus (Aug 2019-Now). Actively performing research has
taught me that very few experiments will work as planned -- problem-solving, flexibility, and persistence are prerequisites to discovery and
success, and when planning, organization and details are crucial. Research has been a prime contributor to my general growth at Loyola
and my journey toward medical school.


\section{Specify concrete mobility plans (research stays) within the FWO fellowship.}
\textit{
Indicate the research stays which are planned within the FWO fellowship. 
Motivate shortly the added value of each stay for the project. Include details on the organization, type of organization, country, contact person, start/end date, function/activities. 
}

I do not currently plan to pursue any research stays at external institutions during the duration of my Ph.D.


\section{List any scientific awards.}
\textit{
List prizes and awards, (e.g. best master thesis…). Specify the awarding body, title, date, amount and theme.
}
Graduate with Bioinformatics Honors?

Alpha Sigma Nu?


\section{Explain how this project fits into the research activities of the involved host institution(s).}
\textit{
Elaborate on the positioning and embedding of your project in the research group(s), its scientific as well as strategic ambitions. 
If applicable, also position your own previous and current research to the proposed PhD fellowship project.
}

IGent -- decentralized web wizards
VITO -- applied technology development + precision health experts


\section{Position the project in a national and international context.}
\textit{
Mention specific research collaborations planned in the course of this project, if appropriate, mention larger projects, programmes or networks your proposal may be part of.
}

TRIPLE Project
Personal Health Data Lockers
GA4GH Beacon initiative


\section{Indicate how the results of the proposed research will be communicated to a non-expert audience.}
\textit{
FWO encourages its fellows to disseminate the results of their research widely and valorise them where possible.
}

Blog? Publications. Public availability with documentation for non-experts. 


\section{Motivate your choice of expert panel.}
\textit{
Carefully read the scientific scope of the selected expert panel and motivate why your application fits the scope of this panel - 
i.e. why this panel has the most appropriate expertise to evaluate your proposal.
}

My project is mainly about implementing cutting edge web and data science technologies to an economically promising use case that has yet to be explored with such technologoes before. 
In this pursuit, the new applications constraints will force innovation and evolution of current state-of-the-art strategies and technologies...


\section{Describe the datatypes (surveys, sequences, manuscripts, objects …) you will collect and/or generate and/or (re)use during your research project.}


\section{Specify in which way the following provisions are in place in order to preserve the data during and at least 5 years after the end of the research.}


\section{Which other issues related to the data management are relevant to mention?}

Not sure? Nothing I can think of?
